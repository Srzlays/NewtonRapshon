%%%%%%%%%%%%%%%%%%%%%%%%%%%%%%%%%%%%%%%%%%%%%%%%%%%%%%%%%%%%%%%%%%%%%%%%%%%
%
% Plantilla para un art�culo en LaTeX en espa�ol.
%
%%%%%%%%%%%%%%%%%%%%%%%%%%%%%%%%%%%%%%%%%%%%%%%%%%%%%%%%%%%%%%%%%%%%%%%%%%%

\documentclass{article}

% Esto es para poder escribir acentos directamente:
\usepackage[latin1]{inputenc}
% Esto es para que el LaTeX sepa que el texto est� en espa�ol:
\usepackage[spanish]{babel}

% Paquetes de la AMS:
\usepackage{amsmath, amsthm, amsfonts}
\usepackage{graphicx}
\usepackage{gensymb}

\usepackage{chicago}

\usepackage[a4paper]{geometry}
\geometry{top=2.5cm, bottom=2.5cm, left=2.5cm, right=2.5cm}
\usepackage{float}
% Teoremas
%--------------------------------------------------------------------------
\newtheorem{thm}{Teorema}[section]
\newtheorem{cor}[thm]{Corolario}
\newtheorem{lem}[thm]{Lema}
\newtheorem{prop}[thm]{Proposici�n}
\theoremstyle{definition}
\newtheorem{defn}[thm]{Definici�n}
\theoremstyle{remark}
\newtheorem{rem}[thm]{Observaci�n}

\renewcommand{\baselinestretch}{1.5}



% Atajos.
% Se pueden definir comandos nuevos para acortar cosas que se usan
% frecuentemente. Como ejemplo, aqu� se definen la R y la Z dobles que
% suelen representar a los conjuntos de n�meros reales y enteros.
%--------------------------------------------------------------------------

\def\RR{\mathbb{R}}
\def\ZZ{\mathbb{Z}}

% De la misma forma se pueden definir comandos con argumentos. Por
% ejemplo, aqu� definimos un comando para escribir el valor absoluto
% de algo m�s f�cilmente.
%--------------------------------------------------------------------------
\newcommand{\abs}[1]{\left\vert#1\right\vert}

% Operadores.
% Los operadores nuevos deben definirse como tales para que aparezcan
% correctamente. Como ejemplo definimos en jacobiano:
%--------------------------------------------------------------------------
\DeclareMathOperator{\Jac}{Jac}

%--------------------------------------------------------------------------
\title{Newton-Rapshon Method Example}
\author{\textit{Srzlays@prtoton.me}\\
}
\date{September 24, 2024}

\begin{document}
\maketitle
\section*{Exersice}
A trough of length $L$ has a cross section in the shape of a semicircle with radius $r$. When filled with water to within a distance $h$ of the top, the volume $V$ of water is 
\begin{equation}
	V= L\left[0.5\pi r^{2} - r^{2}\arcsin\left(\frac{h}{r}\right) - h\left(r^{2} - h^{2}\right)^{1/2}\right],
	\label{eq:OroginalEquation}
\end{equation}
\begin{figure}[H]
	\centering
	\includegraphics[width=0.6\textwidth]{img/img00.png}
	\caption{Graphical representation of the problem}
	\label{fig:TanqueP1}
\end{figure} 
Suppose $L=10\ ft,\ r= 1\ ft,\ $ and $V = 12.4\ ft.\ $ Find the depth of water in the trough. 	This is a problem from chapter 2 of Richard Burden's numerical analysis text\cite{burden2015numerical}.
\section*{Solution}
Note that the equation can be written as:
\begin{equation}
	\arcsin\left(\frac{h}{r}\right) + \frac{h}{r}\left(1-\left(\frac{h}{r}\right)^{2}\right)^{1/2} + \frac{V}{L} - 0.5\pi = 0,
	\label{eq:FactorizedEquation}
\end{equation}
let me do $x = \frac{h}{r},\ $ and we can write the equation as \begin{equation}
	\arcsin(x) + x(1-x^{2})^{1/2} + C = 0,
	\label{eq:SimplifiedEquation}
\end{equation}
where
\begin{equation}
	C = \frac{V}{L} - 0.5\pi.
\end{equation}
We can solve equation \ref{eq:SimplifiedEquation} with Newton-Rapshon method. In this way we can obtain the value of $h,\ $ and thus we can determine the depth of the water which is given by
\begin{eqnarray}
	depth = r - h.
\end{eqnarray}
We found that the water depth is,  $depth = 0.833\ ft$

\bibliographystyle{chicago}
\bibliography{ref}

%-----------------------------------------------------------------
%\begin{thebibliography}{99}

%\bibitem{Cd94} Autor, \emph{T�tulo}, Revista/Editor, (a�o)

%\end{thebibliography}

\end{document}